\zihao{-4}

% \hspace{-0.85cm}院\ \ 系\underline{\makebox[6.7cm]{数学学院}}
% 专\ \ 业\underline{\makebox[6.7cm]{应用数学}}
% \\年\ \ 级\underline{\makebox[6.7cm]{2077级}}
% 姓\ \ 名\underline{\makebox[6.7cm]{张三}}
% \\题\ \ 目\underline{\makebox[14.65cm]{西南交通大学本科生毕业论文(设计)LaTeX模板的制作}}
% \\
% \\
% 指导教师
% \\评\quad\quad 语\uline{\hbox to 13.95cm{}}

% \hspace{0.85cm}\underline{\hbox to 13.95cm{}}

% \hspace{0.85cm}\underline{\hbox to 13.95cm{}}

% \hspace{0.85cm}\underline{\hbox to 13.95cm{}}

% \hspace{0.85cm}\underline{\hbox to 13.95cm{}}

% \hspace{0.85cm}\underline{\hbox to 13.95cm{}}

% \hspace{0.85cm}\underline{\hbox to 13.95cm{}}

% \hspace{0.85cm}\underline{\hbox to 13.95cm{}}

% \hspace{0.85cm}\underline{\hbox to 13.95cm{}}
% \\

% \hspace{9.8cm}指导教师\underline{\hbox to 2cm{}}(签章)
% \\
% \\
% 评\ \ 阅\ \ 人
% \\评\quad\quad 语\underline{\hbox to 13.95cm{}}

% \hspace{0.85cm}\underline{\hbox to 13.95cm{}}

% \hspace{0.85cm}\underline{\hbox to 13.95cm{}}

% \hspace{0.85cm}\underline{\hbox to 13.95cm{}}

% \hspace{0.85cm}\underline{\hbox to 13.95cm{}}

% \hspace{0.85cm}\underline{\hbox to 13.95cm{}}

% \hspace{0.85cm}\underline{\hbox to 13.95cm{}}

% \hspace{0.85cm}\underline{\hbox to 13.95cm{}}

% \hspace{0.85cm}\underline{\hbox to 13.95cm{}}
% \\

% \hspace{9.8cm}评\ \ 阅\ \ 人\underline{\hbox to 2cm{}}(签章)
% \\

% \hspace{-0.85cm}成绩\underline{\hbox to 5.5cm{}}
% \\
% \\答辩委员会主任\underline{\hbox to 2cm{}}(签章)
% \\

% \hfill \hspace{2cm}年\hspace{1cm}月\hspace{1cm}日

% \newpage
\vspace*{0.5cm}
\begin{center}{\hei \zihao{-2} \textbf{城轨供变电课程设计(论文)任务书}}\end{center}
\vspace{1cm}

\hspace{-0.85cm}班\ \ 级\underline{\makebox[3.91cm]{峨眉来的土猴子}}
学生姓名\underline{\makebox[3.91cm]{Hanserwei}}
学\ \ 号\underline{\makebox[3.91cm]{2020114514}}
\\发题日期:\hspace{2cm}年\hspace{1cm}月\hspace{1cm}日\hfill 完成日期:\hspace{2cm}年\hspace{1cm}月\hspace{1cm}日
\\
\\题\ \ 目\underline{\makebox[14.55cm]{轨道交通变电所课程设计}}
\\1、本设计(论文)的目的、意义

\uline{通过该设计,使学生初步掌握轨道交通变电所的设计步骤和方法;熟悉有关设计规范和设计手册的使用;掌握变电所一次设备的选型方法和步骤;掌握变电所二次接线图纸的分析方法;锻炼学生综合运用所学知识的能力,为今后进行工程设计奠定良好的基础。}
\\2、学生应完成的任务\\
\noindent
\uline{1、按给定图纸,分析开闭所、主变电所、牵引/降压混合所、跟随所、降压所主接线电气主接线;\\                                   2、进行短路计算,以及一次设备的选型计算;\\                                                            
	   3、分析二次接线图,包括进线柜、计量柜、馈线柜、母联柜等,分析各回路的工作原理或动作过程;  \\                       4、提交详细的课程设计说明书。
}
\\3、指导教师评语\\
\underline{\hbox to 155mm{}} \\
\underline{\hbox to 155mm{}} \\
\underline{\hbox to 155mm{}} \\
\underline{\hbox to 155mm{}} \\
\vfill

指导教师:\hspace{10cm}年\hspace{1cm}月\hspace{1cm}日