\begin{cnabstract}
随着我国电气化铁路、城市轨道交通,特别是高速铁路的不断发展,新型的牵引供变电系统得到了广泛应用。现代轨道交通电力牵引相较于其他地面交通系统,具有许多优点,如快速高效的运输能力、较低的环境污染、高电力牵引效率,以及易于实现全面信息化和自动化等特点。这些优势提高了劳动生产率和经济效益,为国家的发展带来了重大收益,因此在未来的社会发展中具有巨大的潜力。\par 

在本次课程设计中,我主要根据《青岛-海阳城际(蓝色硅谷段)轨道交通工程、施工设计图》对城轨供变电系统进行了详细分析。分析内容可以分为一次部分和二次部分两大模块。一次部分包括以下三个主要方面:开闭所、主变电所、牵引/降压混合所、跟随所、降压所的主要接线;交流系统的短路电流计算,以进行所选电气设备的动态稳定性和热稳定性验证,并相应进行短路计算;还有设备选型和稳定性校验。设备主要包括断路器、电流互感器、电压互感器、母线、电缆等。设备选型需要考虑其额定电压和额定电流,根据这些参数选择具体的设备型号,然后进行动态和热稳定性校验。\par 

二次部分主要涉及对变电所的各种开关柜接线图进行分析,包括断路器、隔离开关和地刀的倒闸操作,以及各模块电路的功能和工作原理等方面。
\end{cnabstract}
\vspace{1em}\par

\cnkeywords{城市轨道交通;供变电系统;设备选型;一次主接线;二次接线}

\begin{enabstract}
With the continuous development of electrified railways, urban rail transit, especially high-speed railways in China, the new traction and power supply systems have been widely applied. Compared to other ground transportation systems, modern electric traction in rail transit has numerous advantages, such as fast and efficient transportation capacity, lower environmental pollution, high electrical traction efficiency, and ease of achieving comprehensive informatization and automation. These advantages have improved labor productivity and economic benefits, bringing significant gains to the country. Therefore, it has enormous potential in the future of social development.

In this course project, I primarily conducted a detailed analysis of the traction and power supply system for urban rail transit based on the "Qingdao-Haiyang Interurban (Blue Silicon Valley Section) Rail Transit Project, Construction Design." The analysis can be divided into two main modules: the primary part and the secondary part. The primary part includes the following three main aspects: the primary connections of the substations, main substations, traction/step-down hybrid substations, follower substations, and step-down substations. It involves calculating the short-circuit currents of the AC system to verify the dynamic stability and thermal stability of the selected electrical equipment and corresponding short-circuit calculations. Equipment selection includes circuit breakers, current transformers, voltage transformers, busbars, cables, and more. Equipment selection involves considering their rated voltage and current, choosing specific equipment models based on these parameters, and conducting dynamic and thermal stability verifications.

The secondary part mainly involves the analysis of various switchgear connection diagrams in the substations, including the operation of circuit breakers, isolating switches, and ground switches, as well as the functionality and working principles of various circuit modules.
\end{enabstract}
\vspace{1em}\par

\enkeywords{Urban rail transit; power supply system; equipment selection; primary connections; secondary connections.}