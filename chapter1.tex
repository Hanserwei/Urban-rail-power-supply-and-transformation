\chapter{绪论}

\section{变电所设计步骤:}
\begin{enumerate}
	\item 选择供变电工程的主接线。
	\item 根据设备工作的正常条件对设备进行初步选择。
	\item 根据设备工作的异常条件对设备进行校验。
	\item 在完成高压设备的选择设计后,进行配电装置的设计,绘出平面布置图和断面图。
	\item 完成配电装置的设计后,进行防雷系统的设计。
	\item 在完成一次系统的设计后,分别对二次系统的控制、测量和保护部分进行设计,完成对二次设备的选择和校验,绘出二次系统展开图。
	\item 对交流电源、直流电源系统进行设计。
\end{enumerate}
高压设备的设计过程可以概括为以下步骤:选择、校验。如果校验通过,设备被视为合格,否则需要重新选择设备,然后再进行校验。这些步骤旨在确保高压设备在正常和异常工作条件下都能够可靠运行。
\section{电气主接线主要设计原则与步骤}
在电气主接线的设计中,应遵循的主要原则与步骤如下:
\begin{enumerate}
	\item 应以批准的设计任务书为依据,以国家经济建设的方针政策和有关部委的技术政策、技术规范和规程为准则,结合工程具体特点和实际调查掌握的各种基础资料(如电源仅限方式、电力系统资料、变电所选址的有关资料等),进行综合分析和方案研究。
	\item 主接线设计与整个牵引供电系统供电方案、电力系统对电力牵引供电方案密切相关,包括牵引网供电方式(AT或直供)、变电所布点、主变压器接线方式和容量、牵引网电压水平及补偿措施、无功、谐波的综合补偿措施以及直流牵引系统电压等级选择等重大综合技术参数和各种技术要求。
	\item 根据供电系统计算结果提供的上述各种技术参数和有关资料,结合牵引变电所高压进线及其与系统联系(有无功功率通过一次母线)、进线继电保护方式、自动装置与监控二次系统类型、自用电系统,以及城市电牵引线路当前运量和发展规划远景等因素,并全面考虑对主接线的基本要求,做出综合分析和方案比较,以期设计合理的电气主接线。
	\item 新技术的应用对牵引变电所主接线的结构和可靠性等将产生直接影响。例如,采用微机监控自动化系统后,由于控制功能和自动化程度增强,可缩短控制操作和事故处理时间,相对提高了主接线的可靠性;电力电子技术在交、直流牵引系统主设备中的应用,使得主接线电路结构将发生变化。
\end{enumerate}
\section{电气主接线的基本要求}
\begin{enumerate}
	\item 首先应保证电力牵引负确、运输用动力、信号负荷安全、可靠供电的需要和电能质量。牵引负荷和部分动力负荷(如地铁的动力、主要照明和信号电源等)为一级负荷,中断供电将直接造成运输阻塞,甚至造成人员伤亡、设备损坏,进而导致社会生产无法正常进行、人们生活不便、城市秩序混乱等经济损失和政治影响,这更是无法估量的。因此,主接线的接线方式必须保证供电的安全可靠性。由事故造成中断供电的机会越少、影响范围越小、停电时间越短,主接线的安全可靠性就越高。为此,除了应由双回输电线或环形电源供电外,牵引变电所主接线应在接线方式选择或采用其他措施(如自动装置)配合下,保证在电路转换、设备检修和事故处理等情况下供电的可靠性和连续性。对于其他动力负荷和地区负荷,则应根据用户的重要程度(一-般为二级、三级负荷,它们对可靠性要求不同)和具体情况的分析,考虑相应的接线形式和可靠性要求。
	\item 对牵引变电所而言,电压是表征电能质量的基本指标,而与电压水平相关的因素,还包括电压不对称度和谐波含量等,主接线应在变压器接线方式、谐波无功补偿和跳崖方面采取有效地改善电压质量的措施。
	\item 具有必要的运行灵活性,使检修维护方便。运行灵活性是指在系统故障或变电所设备故障和检修时,能适应调度的要求,达到灵活、简便、迅速地倒换运行方式,且故障的影响范围最小。复杂的接线方式对保证操作转换方便显然不利,甚至增大了误操作几率。但过于简单的接线方式,往往不能满足运行方式改变后对可靠供电的要求,增加中断供电时间,这两种情况都应该避免。基于现代技术的自动装备(入备用电源自投)和监控自动化系统的应用,对跳主接线的运行灵活性、可靠性都是有利的,应从变电所整体的全面(一次、二次系统)设计来考虑。
	\item 应有较好的经济性,力求减少投资和运行费用(维修与能耗费)。经济性主要取决于汇流母线的结构类型与组数(几组母线)、主变压器容量、结构形式和数量、高压断路器配电单元数量、配电装备结构类型(屋内或屋外式)和占地面积的因素。经济性往往与可靠性之间存在矛盾,要增强主接线可靠性与灵活性,将导致增加设备和投资。为此,必须力求技术、经济两者统一,在满足供电可靠、运行灵活方便的基础上,尽量使投资和运行费用最省。在可能和充分论证的条件下,可采取按远期规划设计主接线规模、分期实施投资、增加设备等措施,达到最好的经济效益。
	\item 应力求接线简洁明了,并有发展和扩建余地。主接线整体结构和各回路应力求简洁清晰,便于操作运行。同时随着经济建设的高速发展和铁路与城市交通运量的迅速增长,牵引变电所增容、增加馈线和其他内容的改建扩建经常存在,因而电气主接线的设计要留有发展余地,预留最终扩建时主接线发展的电气可能性和场地条件。
\end{enumerate}

